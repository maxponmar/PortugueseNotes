\documentclass[10pt,a4paper]{article}
\usepackage[utf8]{inputenc}
\usepackage[english]{babel}
\usepackage{amsmath}
\usepackage{amsfonts}
\usepackage{amssymb}
\usepackage{soul}
\usepackage[margin=1in]{geometry}
\usepackage{enumitem}
\usepackage{adjustbox}

\newlength{\drop}

\usepackage{tikz}

\newcommand{\inline}[2]{%
    \begin{tikzpicture}[baseline=(word.base), txt/.style={shape=rectangle, inner sep=0pt}]% the baseline key ensures that nodes won't shift up if there's text with descenders, and the txt style removes extra spacing so you can use this inline
    \node[txt] (word) {#1};% the first argument is the contents of the main node
    \node[above] at (word.north) {\footnotesize{#2}};% the second argument is the tag; you can play with the positioning as necessary
    \end{tikzpicture}%
    }


\begin{document}

\begin{titlepage}

\drop=0.1\textheight
    \centering
    \vspace*{\baselineskip}
    \rule{\textwidth}{1.6pt}\vspace*{-\baselineskip}\vspace*{2pt}
    \rule{\textwidth}{0.6pt}\\[\baselineskip]
    {\LARGE PORTUGUESE\\[0.2\baselineskip] NOTES}\\[0.2\baselineskip]
    \rule{\textwidth}{0.4pt}\vspace*{-\baselineskip}\vspace{3.2pt}
    \rule{\textwidth}{1.6pt}\\[\baselineskip]
    \scshape

    \vspace*{2\baselineskip}
    Edited by \\[\baselineskip]
    {\Large MAXIMILIANO PONCE\par}
    {\itshape My Portuguese journey \\\par}
    \vfill
    {\scshape March 2020} \\
    {\large MAXIMILIANO PONCE}\par

\end{titlepage}

\tableofcontents
\newpage

\section{Introduction}
\indent
Olá! Eu sou Maximiliano e sou estudante de engenharia mecatrônica, eu moro no norte do Mexico e gosto muito da língua portugesa. Minha língua materna é espanhol e minha segunda língua é inglês. Atualmente, estoi melhorando meu inglês e estudando português. \\

\indent
Eu espero que este documento possa servir pra outras pessoas que querem aprender a língua portuguesa.\\

\indent
Quero parabenizar Virginia Langhammer, pelo ótimo conteúdo em seu canal no YouTube e por me motivar a estudar português.
\newpage

\end{document}
